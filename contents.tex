\mode*

\section{What's the problem?}

\begin{frame}
  \begin{question}
    \begin{itemize}
      \item We want to have interaction during lectures.
      \item Why do we want this?
    \end{itemize}
  \end{question}
\end{frame}

\begin{frame}
  \begin{block}{Learn by trying, learn by being told?}
    \begin{itemize}
      \item \Textcite{Szekely1950} studied the problem in 1950.
    \end{itemize}
    \begin{enumerate}
      \item<1,3> Presented a puzzling phenomena (Problem A) to a group.
      \item<1,3> Presented a text explaining the phenomena afterwards.
    \end{enumerate}
    \begin{enumerate}
      \item<2,3> Presented a text explaining the phenomena.
      \item<2,3> Presented the puzzling phenomena afterwards.
    \end{enumerate}
    \begin{itemize}
      \item<4> Called them back a week later with related Problem B.
      \item<4> The order for Problem A had significant impact on solving 
        Problem B.
    \end{itemize}
  \end{block}
\end{frame}

\begin{frame}
  \begin{block}{Reproduced}
    \begin{itemize}
      \item Confirmed by later studies as well.
      \item \Textcite{BransfordSchwartz1999} studied effect on future learning.
      \item Students learn more easily from being told if they tried first.
    \end{itemize}
  \end{block}

  \pause

  \begin{remark}
    \begin{itemize}
      \item \emph{Even if the time spent is the same!}
    \end{itemize}
  \end{remark}
\end{frame}

\begin{frame}
  \begin{remark}
    \begin{itemize}
      \item If you want an explanation for this phenomenon, read
    \end{itemize}
    \begin{quote}
      \fullcite{NecessaryConditionsOfLearning}
    \end{quote}
  \end{remark}
\end{frame}

\begin{frame}
  \begin{question}
    \begin{itemize}
      \item What are the obstacles we face?
    \end{itemize}
  \end{question}
\end{frame}

\begin{frame}
  \begin{block}{The problem}
    \begin{itemize}
      \item<1> We want to activate the students.
      \item<1> The students don't want to be 
        activated~\cite{ActualVSFeelingOfLearning}.
      \item<2> The students are scared of the cam and mic.
      \item<2> We need to change culture.
    \end{itemize}
  \end{block}
\end{frame}


\section{How to solve it?}

\begin{frame}
  \begin{block}{The problem}
    \begin{itemize}
      \item The students are scared of the cam and mic.
      \item We need to change culture.
    \end{itemize}
  \end{block}

  \begin{question}
    \begin{itemize}
      \item How can we approach this?
    \end{itemize}
  \end{question}
\end{frame}

\begin{frame}
  \begin{idea}
    \begin{itemize}
      \item Show them it isn't that bad, rather quite the opposite.
    \end{itemize}
  \end{idea}

  \pause

  \begin{solution}[Initial seminar]
    \begin{itemize}
      \item Mandatory seminar reported to LADOK.
      \item Means mandatory webcam (due to ID).
      \item Means mandatory active participation.
    \end{itemize}
  \end{solution}

  \begin{example}[My information and computer security courses]
    \begin{itemize}
      \item Usually 30--40 students.
      \item Could manage on my own.
    \end{itemize}
  \end{example}
\end{frame}

\begin{frame}
  Let's have a look at the design \dots
\end{frame}

\begin{frame}
  \begin{solution}[Ice-breaking question]
    \begin{enumerate}
      \item What's your favourite food?
      \item What's the best Christmas gift you've given/gotten?
      \item Best covid-19-time survival tips?
      \item Best spring activity now that snow is gone?
    \end{enumerate}
  \end{solution}

  \begin{example}[Kompletterande perspektiv för indek]
    \begin{itemize}
      \item Roughly 10--15 students (per group).
      \item Suitable size to not take too much time (\(< 10\) min).
      \item Go through participants by name to save time.
    \end{itemize}
  \end{example}
\end{frame}

\begin{frame}
  \begin{block}{The problem}
    \begin{itemize}
      \item We want to activate the students.
      \item The students don't want to be activated, they want to lean back and 
        enjoy a show~\cite{ActualVSFeelingOfLearning}.
      \item We need to change culture.
    \end{itemize}
  \end{block}

  \begin{question}
    \begin{itemize}
      \item Above we had small-group seminars.
      \item How to get to lectures?
    \end{itemize}
  \end{question}
\end{frame}

\begin{frame}
  \begin{solution}[Size]
    \begin{itemize}
      \item Divide them into breakout rooms.
    \end{itemize}
  \end{solution}
\end{frame}

\begin{frame}
  \begin{solution}[Work to remove fear]
    \begin{itemize}
      \item Ice-breaker questions in breakout rooms.
      \item Three-by-three.
      \item Repeat a few times.
    \end{itemize}
  \end{solution}

  \pause

  \begin{solution}[Change culture]
    \begin{itemize}
      \item Be active.
      \item Be active.
      \item \dots
      \item Be active.
    \end{itemize}
  \end{solution}
\end{frame}

\begin{frame}
  \begin{solution}[Flipped classroom]
    \begin{itemize}
      \item The students watch videos in advance.
      \item Make videos interactive with something like FeedbackFruits.
      \item Then we work on a problem in class.
    \end{itemize}
  \end{solution}

  \begin{example}[DD2350 Applied Crypto]
    \begin{itemize}
      \item Roughly 40 students.
      \item Break up into breakout rooms.
      \item Manageable alone.
    \end{itemize}
  \end{example}
\end{frame}

\begin{frame}
  Let's have a look at the design \dots
\end{frame}

\begin{frame}
  \begin{question}
    \begin{itemize}
      \item To be this active works with classes up to 40--60 students.
      \item What to do when we have 150--200 students?
    \end{itemize}
  \end{question}
\end{frame}

\begin{frame}
  \begin{remark}
    \begin{itemize}
      \item This still works for these sizes too.
      \item It's just impossible to go through all breakout rooms.
      \item But you don't have too.
      \item Mark Decker (KTH SOTL'15 keynote) did exactly this, but on campus.
    \end{itemize}
  \end{remark}
\end{frame}

\begin{frame}
  \begin{remark}[One problem remains \dots]
    \begin{itemize}
      \item Students can be quite active in the chat.
    \end{itemize}
  \end{remark}

  \pause

  \begin{solution}[Co-pilot]
    \begin{itemize}
      \item Students active in the chat during lecture.
      \item Co-pilot keeps track of chat.
      \item Co-pilot interacts with the lecturer.
      \item Paper by Bosk and Glassey in KTH SOTL'21.
    \end{itemize}
  \end{solution}
\end{frame}

\begin{frame}
  \begin{example}[DD1301 Computer intro]
    \begin{itemize}
      \item Roughly 260 students.
      \item 300+ messages during an hour!
      \item I had Ric.
    \end{itemize}
  \end{example}

  \pause

  \begin{example}[DD1337 Programming (data)]
    \begin{itemize}
      \item Roughly 200 students.
      \item Hard to manage alone.
      \item Ric had me.
    \end{itemize}
  \end{example}
\end{frame}

\begin{frame}
  \begin{example}[DD1315 Programming (indek)]
    \begin{itemize}
      \item Roughly 160 students.
      \item Hard to manage alone.
      \item I had a TA.
    \end{itemize}
  \end{example}
\end{frame}

\begin{frame}
  \begin{question}
    \begin{itemize}
      \item What other approaches have you encountered?
      \item During the discussions in the beginning?
    \end{itemize}
  \end{question}
\end{frame}

\section{Take away}

\begin{frame}
  \begin{block}{Work to remove fears}
    \begin{itemize}
      \item Show it's not that bad.
      \item Do it already in the beginning.
    \end{itemize}
  \end{block}

  \pause

  \begin{block}{Change culture}
    \begin{itemize}
      \item Be consistent throughout the course (programme).
      \item The beginning is crucial.
      \item But we must keep it up.
    \end{itemize}
  \end{block}
\end{frame}
